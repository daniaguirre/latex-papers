\subsection{General operation}

\begin{itemize}
    \item Use frequency of dynamic links during the exploration phase ($f_e$).
    \item Visit frequency to candidate nodes during the exploration phase ($f_n$).
\end{itemize}

\subsection{Initial conditions}

\subsubsection{Initial topology} 
The initial topology is given by a 2D grid, where each node has a Von Neumann neighborhood consisting of the top, bottom, left and right neighbors, except for the nodes lying on the corners and borders of the 2D grid, which have 2 or 3 neighbors, respectively. Links belonging to this initial topology are fixed along the rewiring mechanism. 

In addition to fixed links, each node has a determinate number of dynamic links. Initially, these dynamic links are only connected to their owning nodes, and can be understood as loose ends. The loose end of each dynamic link is rewired along the rewiring mechanism, while the end connected to its owning node remains fixed. 

It is important to mention that we proposed a 2D grid as initial topology because the rewiring mechanism is designed to work on real telecommunication networks, where the length of physical links is a restriction that must be take into account. Since the 2D grid has a natural embedding on the euclidean space it is easy to compute 

\subsubsection{Compass routing}

The rewiring mechanism requires a routing algorithm to explore the network and exchange messages between nodes. Since 
The compass routing algorithm was designed to work on planar graph



we want to travel from an initial vertex s to a destination vertex t , and that all the information available to us at any point in time is: the coordinates of our destination, our current position, and the directions of the edges incident with the vertex at which we are located. Starting at s , we will in a recursive way choose to traverse the edge of the geometric graph incident to our current position and with the closest
slope to that of the line segment connecting the vertex of our current position to t. Ties are broken randomly.

To choose the next step of a packet, the routing is performed as follows: we draw a main straight line between the current node in charge and the final destination. Also, we draw
a secondary line between the node in charge and each of its immediate neighbors. Each secondary
line crosses the main line with a different angle. The neighbor corresponding to the
smallest angle will be designated as the next node in the route of the tracer

Each node is identified by a coordinate in the initial 2D grid as is shown in Fig. \ref{label}.
The Compass routing algorithm was designed to on planar graphs

\subsubsection{Coordinator election}

As is explained in the following section, the coordinator node have two tasks: 1) to count the number of cycles of the rewiring mechanism, and 2) to coordinate the start and end of each cycle phase.



\subsection{Rewiring mechanism}

The rewiring mechanism is a distributed process that is executed in a determined number of cycles. The coordinator node starts each cycle spreading notification messages through a PI algorithm. Each cycle consists of the following four phases:

\subsubsection{Exploration}

Each node starts its exploration phase when it receives a notification from the coordinator node. During this phase, each node sequentially sends twenty tracer packets to arbitrary destination nodes. Packets are routed applying compass routing. Destination nodes respond to a tracer packet with an acknowledgment packet containing the path followed by the tracer packet.
When a node receives an acknowledgment packet, it updates its vectors $f_e$ and $f_n$. A node finishes its exploration phase when it has received acknowledgments packets for its twenty tracer packets.

\subsubsection{Exploration synchronization}

This phase starts when the coordinator node has finished its exploration phase. Then it starts a PIF algorithm to spread messages asking if the rest of nodes have finished its exploration phase. When the coordinator node receives an acknowledgment from all the nodes in the network,  it starts a PI algorithm to spread messages notifying that nodes must start its rewiring phase.

\subsubsection{Rewiring}

Each node starts its rewiring phase when it receives a notification from the coordinator. Then, the notified node uses the information in its vectors $f_e$ and $f_n$ to take a rewiring decision according to one of the following rules:
\begin{itemize}
    \item \textbf{Rule 1}. A node rewires its least used dynamic link (according with $f_e$) to the most visited node (according to $f_n$).
    \item \textbf{Rule 2}. A node rewires its least used dynamic link (according with $f_e$) to the first node at distance 2 from it, which is the first node in $f_n$.
\end{itemize}

\subsubsection{Rewiring synchronization}

Analogous to the \textit{Exploration Synchronization phase}, this phase starts when the coordinator node has finished its rewiring phase. Then it starts a PIF algorithm to spread messages asking if the rest of nodes have finished its rewiring phase. When the coordinator node receives an acknowledgment from all the nodes in network, it checks if all cycles have been completed. If so, the rewiring mechanism finishes. Otherwise, the coordinator node updates the cycle counter and starts a PI algorithm to spread messages notifying that the cycle has finished and nodes must start the \textit{Exploration phase} of a new cycle.
