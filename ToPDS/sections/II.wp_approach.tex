\label{sec:2-WPapproach}

This section explains how paths and nodes in a CG can be represented as words of a regular language; and how this representation was applied for developing path computation algorithms in CGs \cite{AGUIRREGUERRERO2019218}. These algorithms are a key component of the WPR.
Before presenting the word-processing approach, it is necessary to give the formal definition of CGs. 

Let $\mathcal{G}=<S|R>$ be an algebraic group, where $S$ and $R$ are the set of generators and relators, respectively \cite[Section 2.2]{group_theory}. Then $\mathcal{G}$ has an associated CG, denoted by $\Gamma(\mathcal{G},S)$, where the set of vertices $V(\Gamma)$ is given by the set of group elements, and there is an edge $e\in E(\Gamma)$ from $g$ to $h$ if and only if  $g\cdot s = h$ for $g,h\in \mathcal{G}$ and $s\in S$. 

\begin{figure*}[ht]
\begin{minipage}{\textwidth} 
\centering

\subfloat[Nodes represent group elements in $Sym_4$ and edges represent generators in $S$. Each generator $(2134)$, $(1324)$ and $(1243)$ is represented by dotted, solid and dashed links respectively.]{\label{fig:bs4}{\begin{tikzpicture}[line width=0.25mm, line cap=round,line join=round,>=triangle 45,x=1.0cm,y=1.0cm]
\begin{scriptsize}

%*************** edges de la grafica *********************************
% e - a
\draw[dotted] (0,5)-- (5,5) node [snow,midway,above] {$(2134)$};
% c - ac
\draw[dotted] (0,0)-- (5,0);

% cb - cba
\draw[dotted] (0.5,0.5)-- (1.5,0.5);
% bcb - bcba
\draw[dotted] (0.5,1.5)-- (1.5,1.5);
% bc - bac
\draw[dotted] (0.5,3.5)-- (1.5,3.5);
% b - ba
\draw[dotted] (0.5,4.5)-- (1.5,4.5);

% acb - acba
\draw[dotted] (3.5,0.5)-- (4.5,0.5);
% abcb - abcba
\draw[dotted] (3.5,1.5)-- (4.5,1.5);
% abc - abac
\draw[dotted] (3.5,3.5)-- (4.5,3.5);
% ab - aba
\draw[dotted] (3.5,4.5)-- (4.5,4.5);

% bacba - bacb
\draw[dotted] (2,2)-- (2,3);
% abacba - abacb
\draw[dotted] (3,2)-- (3,3);

% c - e_A
\draw[dashed] (0,0)-- (0,5) node [snow,midway,left] {$(1243)$};
% a - ac
\draw[dashed] (5,0)-- (5,5);

% acb - abcb
\draw[dashed] (0.5,0.5)-- (0.5,1.5);
% abc - ab
\draw[dashed] (0.5,3.5)-- (0.5,4.5);
% acba - abcba
\draw[dashed] (1.5,0.5)-- (1.5,1.5);
% abac - aba
\draw[dashed] (1.5,3.5)-- (1.5,4.5);
% cba - bcba
\draw[dashed] (3.5,0.5)-- (3.5,1.5);
% bac - ba
\draw[dashed] (3.5,3.5)-- (3.5,4.5);
% cb - bcb
\draw[dashed] (4.5,0.5)-- (4.5,1.5);
% bc - b
\draw[dashed] (4.5,3.5)-- (4.5,4.5);

% bacba - abacba
\draw[dashed] (2,2)-- (3,2);
% bacb - abacb
\draw[dashed] (2,3)-- (3,3);

% e - b
\draw[] (0,5)-- (0.5,4.5) node [snow,midway,right] {$(1324)$};
% bac - bacb
\draw[] (1.5,3.5)-- (2,3);
% bacba - bcba
\draw[] (3,2)-- (3.5,1.5);

% c - cb
\draw[] (0,0)-- (0.5,0.5);
% bcba - bacba
\draw[] (1.5,1.5)-- (2,2);
% abacb - abac
\draw[] (3,3)-- (3.5,3.5);

% a - ab
\draw[] (5,5)-- (4.5,4.5);
% ac - acb
\draw[] (5,0)-- (4.5,0.5);

% cba - acba
\draw[] (1.5,0.5)-- (3.5,0.5);
% ba - aba
\draw[] (1.5,4.5)-- (3.5,4.5);
% bcb - bc
\draw[] (0.5,1.5)-- (0.5,3.5);
% abcb - abc
\draw[] (4.5,1.5)-- (4.5,3.5);

%*************** nodos de la grafica *********************************

% e_A - 1234
\fill [color=gold] (0,5) circle (2.5pt);
\draw[color=snow] (0.2,5.2) node {1234};

% a - 2134
\fill [color=gold] (5,5) circle (2.5pt);
\draw[color=snow] (4.8,5.2) node {2134};

% c - 1243
\fill [color=gold] (0,0) circle (2.5pt);
\draw[color=snow] (0.2,-0.2) node {1243};

% ac - 2143
\fill [color=gold] (5,0) circle (2.5pt);
\draw[color=snow] (4.8,-0.2) node {2143};

%%%%%%%%%%%%%%%%%%%%%%%%%%%%%%%%%%%%%%%%%

% b - 1324
\fill [color=gold] (0.5,4.5) circle (2.5pt);
\draw[color=snow] (0.9,4.3) node {1324};

% ab - 2314
\fill [color=gold] (4.5,4.5) circle (2.5pt);
\draw[color=snow] (4.3,4.7) node {2314};

% cb - 1423
\fill [color=gold] (0.5,0.5) circle (2.5pt);
\draw[color=snow] (0.7,0.3) node {1423};

% acb - 2413
\fill [color=gold] (4.5,0.5) circle (2.5pt);
\draw[color=snow] (4.3,0.3) node {2413};

%%%%%%%%%%%%%%%%%%%%%%%%%%%%%%%%%%%%%%%%%

% ba - 3142
\fill [color=gold] (1.5,4.5) circle (2.5pt);
\draw[color=snow] (1.9,4.3) node {3124};

% aba - 3214
\fill [color=gold] (3.5,4.5) circle (2.5pt);
\draw[color=snow] (3.1,4.3) node {3214};

% cba - 4123
\fill [color=gold] (1.5,0.5) circle (2.5pt);
\draw[color=snow] (1.9,0.7) node {4123};

% acba - 4213
\fill [color=gold] (3.5,0.5) circle (2.5pt);
\draw[color=snow] (3.1,0.7) node {4213};

%%%%%%%%%%%%%%%%%%%%%%%%%%%%%%%%%%%%%%%%%

% bc - 1324
\fill [color=gold] (0.5,3.5) circle (2.5pt);
\draw[color=snow] (0.9,3.3) node {1342};

% abc - 2341
\fill [color=gold] (4.5,3.5) circle (2.5pt);
\draw[color=snow] (4.1,3.3) node {2341};

% bcb - 1423
\fill [color=gold] (0.5,1.5) circle (2.5pt);
\draw[color=snow] (0.9,1.7) node {1432};

% abcb - 2413
\fill [color=gold] (4.5,1.5) circle (2.5pt);
\draw[color=snow] (4.1,1.7) node {2431};

%%%%%%%%%%%%%%%%%%%%%%%%%%%%%%%%%%%%%%%%%

% bac - 3124
\fill [color=gold] (1.5,3.5) circle (2.5pt);
\draw[color=snow] (1.9,3.7) node {3142};

% abac - 3241
\fill [color=gold] (3.5,3.5) circle (2.5pt);
\draw[color=snow] (3.1,3.7) node {3241};

% bcba - 4132
\fill [color=gold] (1.5,1.5) circle (2.5pt);
\draw[color=snow] (1.9,1.3) node {4132};

% abcba - 4231
\fill [color=gold] (3.5,1.5) circle (2.5pt);
\draw[color=snow] (3.1,1.3) node {4231};

%%%%%%%%%%%%%%%%%%%%%%%%%%%%%%%%%%%%%%%%%

% bacb - 3421
\fill [color=gold] (3,3) circle (2.5pt);
\draw[color=snow] (3.4,2.8) node {3421};

% abacb - 3412
\fill [color=gold] (2,3) circle (2.5pt);
\draw[color=snow] (1.6,2.8) node {3412};

% bacba - 4312
\fill [color=gold] (3,2) circle (2.5pt);
\draw[color=snow] (3.4,2.2) node {4312};

% abacba - 4321
\fill [color=gold] (2,2) circle (2.5pt);
\draw[color=snow] (1.6,2.2) node {4321};

\end{scriptsize}
\end{tikzpicture}}}
\hspace{0.05\textwidth}
\subfloat[Edges labelled with letters representing their corresponding generators according with Eq. (\ref{eq:mapGenAlpBS}).]{\label{fig:bs4:edgesLab}{\begin{tikzpicture}[line width=0.25mm, line cap=round,line join=round,>=triangle 45,x=1.0cm,y=1.0cm]
\begin{scriptsize}

%*************** edges de la grafica *********************************
% e - a
\draw[] (0,5)-- (5,5) node [snow,midway,above] {$a$};
% c - ac
\draw[] (0,0)-- (5,0) node [snow,midway,below] {$a$};
% cb - cba
\draw[] (0.5,0.5)-- (1.5,0.5) node [snow,midway,above] {$a$};
% bcb - bcba
\draw[] (0.5,1.5)-- (1.5,1.5) node [snow,midway,below] {$a$};
% bc - bac
\draw[] (0.5,3.5)-- (1.5,3.5) node [snow,midway,above] {$a$};
% b - ba
\draw[] (0.5,4.5)-- (1.5,4.5) node [snow,midway,below] {$a$};
% acb - acba
\draw[] (3.5,0.5)-- (4.5,0.5) node [snow,midway,above] {$a$};
% abcb - abcba
\draw[] (3.5,1.5)-- (4.5,1.5) node [snow,midway,below] {$a$};
% abc - abac
\draw[] (3.5,3.5)-- (4.5,3.5) node [snow,midway,above] {$a$};
% ab - aba
\draw[] (3.5,4.5)-- (4.5,4.5)node [snow,midway,below] {$a$};
% bacba - bacb
\draw[] (2,2)-- (2,3) node [snow,midway,right] {$a$};
% abacba - abacb
\draw[] (3,2)-- (3,3)node [snow,midway,left] {$a$};

% c - e_A
\draw[] (0,0)-- (0,5) node [snow,midway,left] {$c$};
% a - ac
\draw[] (5,0)-- (5,5) node [snow,midway,right] {$c$};

% 
\draw[] (0.5,0.5)-- (0.5,1.5) node [snow,midway,right] {$c$};
% abc - ab
\draw[] (0.5,3.5)-- (0.5,4.5) node [snow,midway,right] {$c$};
% acba - abcba
\draw[] (1.5,0.5)-- (1.5,1.5) node [snow,midway,left] {$c$};
% bac - ba
\draw[] (1.5,3.5)-- (1.5,4.5) node [snow,midway,left] {$c$};
% acb - abcb
\draw[] (3.5,0.5)-- (3.5,1.5) node [snow,midway,right] {$c$};
% aba - abac
\draw[] (3.5,3.5)-- (3.5,4.5) node [snow,midway,right] {$c$};
% acb - acbc
\draw[] (4.5,0.5)-- (4.5,1.5) node [snow,midway,left] {$c$};
% abc - ab
\draw[] (4.5,3.5)-- (4.5,4.5) node [snow,midway,left] {$c$};

% bacba - abacba
\draw[] (2,2)-- (3,2) node [snow,midway,above] {$c$};
% bacb - abacb
\draw[] (2,3)-- (3,3) node [snow,midway,below] {$c$};

% e - b
\draw[] (0,5)-- (0.5,4.5) node [snow,near end,left] {$b$};
% bac - bacb
\draw[] (1.5,3.5)-- (2,3) node [snow,near end,above] {$b$};
% bacba - bcba
\draw[] (3,2)-- (3.5,1.5) node [snow,near end,above] {$b$};
% c - cb
\draw[] (0,0)-- (0.5,0.5) node [snow,midway,above] {$b$};
% bcba - bacba
\draw[] (1.5,1.5)-- (2,2) node [snow,midway,above] {$b$};
% abacb - abac
\draw[] (3,3)-- (3.5,3.5) node [snow,near start,above] {$b$};
% a - ab
\draw[] (5,5)-- (4.5,4.5) node [snow,midway,below] {$b$};
% ac - acb
\draw[] (5,0)-- (4.5,0.5) node [snow,midway,above] {$b$};
% cba - acba
\draw[] (1.5,0.5)-- (3.5,0.5) node [snow,midway,above] {$b$};
% ba - aba
\draw[] (1.5,4.5)-- (3.5,4.5) node [snow,midway,below] {$b$};
% bcb - bc
\draw[] (0.5,1.5)-- (0.5,3.5) node [snow,midway,right] {$b$};
% abcb - abc
\draw[] (4.5,1.5)-- (4.5,3.5) node [snow,midway,left] {$b$};

%*************** nodos de la grafica *********************************

% e_A - 1234
\fill [color=gold] (0,5) circle (2.5pt);
\draw[color=snow] (0.2,5.2) node {1234};

% a - 2134
\fill [color=gold] (5,5) circle (2.5pt);
\draw[color=snow] (4.8,5.2) node {2134};

% c - 1243
\fill [color=gold] (0,0) circle (2.5pt);
\draw[color=snow] (0.2,-0.2) node {1243};

% ac - 2143
\fill [color=gold] (5,0) circle (2.5pt);
\draw[color=snow] (4.8,-0.2) node {2143};

%%%%%%%%%%%%%%%%%%%%%%%%%%%%%%%%%%%%%%%%%

% b - 1324
\fill [color=gold] (0.5,4.5) circle (2.5pt);
\draw[color=snow] (0.9,4.7) node {1324};

% ab - 2314
\fill [color=gold] (4.5,4.5) circle (2.5pt);
\draw[color=snow] (4.3,4.7) node {2314};

% cb - 1423
\fill [color=gold] (0.5,0.5) circle (2.5pt);
\draw[color=snow] (0.7,0.3) node {1423};

% acb - 2413
\fill [color=gold] (4.5,0.5) circle (2.5pt);
\draw[color=snow] (4.3,0.3) node {2413};

%%%%%%%%%%%%%%%%%%%%%%%%%%%%%%%%%%%%%%%%%

% ba - 3142
\fill [color=gold] (1.5,4.5) circle (2.5pt);
\draw[color=snow] (1.9,4.3) node {3124};

% aba - 3214
\fill [color=gold] (3.5,4.5) circle (2.5pt);
\draw[color=snow] (3.1,4.3) node {3214};

% cba - 4123
\fill [color=gold] (1.5,0.5) circle (2.5pt);
\draw[color=snow] (1.9,0.7) node {4123};

% acba - 4213
\fill [color=gold] (3.5,0.5) circle (2.5pt);
\draw[color=snow] (3.1,0.7) node {4213};

%%%%%%%%%%%%%%%%%%%%%%%%%%%%%%%%%%%%%%%%%

% bc - 1324
\fill [color=gold] (0.5,3.5) circle (2.5pt);
\draw[color=snow] (0.9,3.3) node {1342};

% abc - 2341
\fill [color=gold] (4.5,3.5) circle (2.5pt);
\draw[color=snow] (4.1,3.3) node {2341};

% bcb - 1423
\fill [color=gold] (0.5,1.5) circle (2.5pt);
\draw[color=snow] (0.9,1.7) node {1432};

% abcb - 2413
\fill [color=gold] (4.5,1.5) circle (2.5pt);
\draw[color=snow] (4.1,1.7) node {2431};

%%%%%%%%%%%%%%%%%%%%%%%%%%%%%%%%%%%%%%%%%

% bac - 3124
\fill [color=gold] (1.5,3.5) circle (2.5pt);
\draw[color=snow] (1.9,3.7) node {3142};

% abac - 3241
\fill [color=gold] (3.5,3.5) circle (2.5pt);
\draw[color=snow] (3.1,3.7) node {3241};

% bcba - 4132
\fill [color=gold] (1.5,1.5) circle (2.5pt);
\draw[color=snow] (1.9,1.3) node {4132};

% abcba - 4231
\fill [color=gold] (3.5,1.5) circle (2.5pt);
\draw[color=snow] (3.1,1.3) node {4231};

%%%%%%%%%%%%%%%%%%%%%%%%%%%%%%%%%%%%%%%%%

% bacb - 3421
\fill [color=gold] (3,3) circle (2.5pt);
\draw[color=snow] (3.4,2.8) node {3421};

% abacb - 3412
\fill [color=gold] (2,3) circle (2.5pt);
\draw[color=snow] (1.6,2.8) node {3412};

% bacba - 4312
\fill [color=gold] (3,2) circle (2.5pt);
\draw[color=snow] (3.4,2.2) node {4312};

% abacba - 4321
\fill [color=gold] (2,2) circle (2.5pt);
\draw[color=snow] (1.6,2.2) node {4321};

\end{scriptsize}
\end{tikzpicture}}}
\caption{Cayley graph of the symmetric group $Sym_4$ with generators $S=\{(2134), (1324),(1243)\}$. This graph is called Bubble-sort graph and denoted by $BS(4)$.}
\label{fig:bs4_full}
\end{minipage}
\end{figure*}

Hereafter vertices in $V(\Gamma)$ and elements of $\mathcal{G}$ are used interchangeably, and likewise links in $E(\Gamma)$ and generators in $S$. For instance, consider the symmetric group $Sym                      _p$,  where the group elements are given by the permutations of the set $\{1,\ldots,p\}$, and the group operation is the composition of permutations. Figure \ref{fig:bs4} presents the CG of $Sym_4$ with the group generators are given by the set of permutations $S=\{(2134)$, $(1324)$, $(1243)\}$, which is known as Bubble-sort graph and denoted by $BS(4)$. Note that nodes $1234$ and $2134$ are connected through the link $(2134)$ due to $2134$ results of apply the permutation $(2134)$ to $1234$ and vice versa.

\subsection{Letters as links and words as paths}
\label{sec:words_as_paths}

Assume that edges of a CG are labelled according to a bijective map
\begin{equation}
\label{eq:bijective_map}
  \phi: \{S \cup S^{-1}\}\to \mathcal{A},
\end{equation}
where $S^{-1}$ is the set of inverses of $S$ and $\mathcal{A}$ is an alphabet. Then $\phi$ assigns each generator and its inverse to lowercase and uppercase variants of the same letter. 
Thus every path in the CG can be represented by a unique word $w$ over $\mathcal{A}$. 
Returning to $BS(4)$, where $S=S^{-1}$, let $\mathcal{A}=\{a,b,c\}$ be an alphabet and let $\phi$ denote the map:
\begin{equation}
\label{eq:mapGenAlpBS}
\begin{matrix}
\phi:& S & \to & \mathcal{A} \\
 & (2134) & \to & a \\ 
 & (1324) & \to & b \\ 
 & (1243) & \to & c 
\end{matrix}
\end{equation}

Figure \ref{fig:bs4:edgesLab} shows $BS(4)$ with their edges labelled according to a map (\ref{eq:bijective_map}). Then, the word $abc$ represents the sequence of edges $(2134)(1324)(1243)$ and thus a path, e.g. between $1234$ to $2341$.

Let $v$ and $w$ be words over $\mathcal{A}$, such that $S$ and $\mathcal{A}$ satisfy map (\ref{eq:mapGenAlpBS}). It has that:
\begin{itemize}
\item The words $v$ and $w$ are called \textbf{equivalent}, i.e. $v=_\mathcal{G}w$, if they represent paths between the same pair of nodes. In Fig. \ref{fig:bs4:edgesLab}, $abc=_\mathcal{G}babca$ due to they represents paths between the same nodes, e.g. $1243$ and $2431$.
\item The \textbf{inverse of $w$}, denoted by $w^{-1}$, is given by the reverse string of the inverse letters of $w$, e.g. if $a^{-1}=A$, $c^{-1}=C$ and $w=aC$, then $w^{-1}=cA$. Consider now that $w$ represents a path from $g$ to $h$ then $w^{-1}$ represents a path from $h$ to $g$. In the case of $BS(4)$, $abc$ represents a path from $1432$ to $4312$, meanwhile $(abc)^{-1}=cba$ represents a path from $4312$ to $1432$.

\item The \textbf{reduced form of $w$}, denoted by $w_{red}$, results from removing the substrings of the form $uu^{-1}$ from $w$. The words $w$ and $w_{red}$ are equivalent, i.e. $w=_\mathcal{G}w_{red}$, and thus represent paths between the same pair of nodes. 
Consider a path $w=abcaaca$  in Fig. \ref{fig:bs4:edgesLab}, then $w_{red}=aba$ due to $(ca)^{-1}=ac$. Since $abcaaca=_\mathcal{G}aba$, these words represent paths between the same pair of nodes, e.g. $1342$ and $4321$.
\end{itemize}

\begin{definition}
Let $\Gamma(\mathcal{G},S)$ be a CG and $\mathcal{A}$ be an alphabet, such that $S$ and $\mathcal{A}$ satisfy map (\ref{eq:bijective_map}). Then, the \textbf{free group} over $\mathcal{A}$, denoted by $\mathcal{F}(\mathcal{A})$, consists of all reduced words over $\mathcal{A}$ including the symbol $e_\mathcal{A}$ that
denotes the null string. 
\end{definition}

From the above definition, the language $\mathcal{F}(\mathcal{A})$ defines all paths in $\Gamma(\mathcal{G}, S)$. In particular, $e_\mathcal{A}$ denotes the empty path. It is also possible to define a language for the shortest paths as follows.

\begin{definition}
\label{def:shortlexL}
Let $<_\mathcal{A}$ be a \textbf{lexicographical order} over $\mathcal{A}$. Let $w$ and $v$ be words over $\mathcal{A}$, it is said that $w$ is \textit{shortLex} than $v$, if $w$ is shorter than $v$, i.e. $|w|<|v|$, or $w$ and $v$ have the same length but $w$ comes before $v$ in the order $<_\mathcal{A}$.
 The \textbf{language of the \textit{shortLex} words} in $\mathcal{F}(\mathcal{A})$ representing a unique group element in $\mathcal{G}$ is given by 
 \begin{equation}
 \label{eq:lan}
 L=\{w\in \mathcal{F}(\mathcal{A}): w<_\mathcal{A} v\textnormal{, }\forall v\in \mathcal{F}(\mathcal{A}) \textnormal{ s.t. } w=_{\mathcal{G}}v\}. 
 \end{equation}
\end{definition}

Language $L$ gives a unique representation for 
the shortest paths  in $\Gamma(\mathcal{G},S)$. 
Continuing with $BS(4)$, let $a<b<c$ be a lexicographic order over the alphabet $\mathcal{A}=\{a,b,c\}$, then the language of the \textit{shortLex} words of $BS(4)$ is

\begin{equation}
     \begin{split}
    \label{eq:lan-B4}
    L =\{&e_\mathcal{A}, a, b, c, ab, ac, ba, bc, cb, aba, abc,acb,  \\
    &bac, bcb, cba,abac, abcb,acba, bacb,\\
    & bcba, abacb, abcba, bacba, abacba\}.
    \end{split}
\end{equation}

Equation (\ref{eq:lan-B4}) defines the shortest paths between each pair of nodes in $BS(4)$. These paths are called \textbf{\textit{shortLex} paths} due to they are represented by the \textit{shortLex} words. For instance, $ac$ and $ca$ represents paths between the same pair of nodes in $BS(4)$, e.g. $4213$ and $2431$. Although both paths are the shortest ones, the \textit{shortLex} path is $ac\in L$ due to $ac<_\mathcal{A} ca$. Hereafter, a path represented by a word $w$ will be denoted as $\widehat{w}$.

\subsection{Words as nodes}

In addition to paths, words in $L$ also give a unique representation for nodes of its related CG\footnote{For every CG, $|V(\Gamma)|=|L|$.}. This representation proceeds as follows. First, the symbol $e_\mathcal{A}$ is assigned to an arbitrary node in $V(\Gamma)$. Then, each of the remaining nodes is assigned to the word in $L$ representing the \textit{shortLex} path from $e_\mathcal{A}$ to it. From now on, a node represented by a word $w$ will be denoted as $\overline{w}$. Figure \ref{fig:BS4_nodeEdgeLabelled} shows $BS(4)$, where nodes are labelled with its corresponding word in $L$, see Eq. (\ref{eq:lan-B4}).

\begin{figure}
    \centering
    \begin{tikzpicture}[line width=0.25mm, line cap=round,line join=round,>=triangle 45,x=1.0cm,y=1.0cm]
\begin{scriptsize}
%*************** grafica 2 ***************%

%*************** edges de la grafica *********************************
% GENERATOR a
% e - a
\draw[color=snow] (0,5) -- (4.9,5) node [snow,midway,above] {$a$};
% c - ac
\draw[color=snow] (0,0)-- (5,0) node [midway,below] {$a$};

% cb - cba
\draw[color=snow] (0.5,0.5)-- (1.5,0.5) node [midway,above] {$a$};
% bcb - bcba
\draw[color=snow] (0.5,1.5)-- (1.5,1.5) node [midway,below] {$a$};
% bc - bac
\draw[color=snow] (0.6,3.5) -- (1.4,3.5) node [midway,above,color=snow] {$a$};
% b - ba
\draw[color=snow] (0.5,4.5)-- (1.5,4.5) node [midway,below] {$a$};

% acb - acba
\draw[color=snow] (3.5,0.5)-- (4.5,0.5) node [midway,above] {$a$};
% abcb - abcba
\draw[color=snow] (3.5,1.5)-- (4.5,1.5) node [midway,below] {$a$};
% abc - abac
\draw[color=snow] (3.5,3.5) -- (4.4,3.5) node [midway,above,color=snow] {$a$};
% ab - aba
\draw[color=snow] (3.5,4.5)-- (4.5,4.5) node [midway,below] {$a$};
% bacba - bacb
\draw[color=snow] (2,2)-- (2,2.9) node [midway,right,color=snow] {$a$};
% abacba - abacb
\draw[color=snow] (3,2)-- (3,3) node [midway,left] {$a$};

% c - e_A
\draw[color=snow] (0,0)-- (0,5) node [midway,left] {$c$};
% a - ac
\draw[color=snow] (5,0)-- (5,5) node [midway,right] {$c$};
% GENERATOR c
% cb - bcb
\draw[color=snow] (0.5,0.5)-- (0.5,1.5) node [midway,right] {$c$};
% bc - b
\draw[color=snow] (0.5,3.5)-- (0.5,4.4) node [midway,right,color=snow] {$c$};
% cba - bcba
\draw[color=snow] (1.5,0.5)-- (1.5,1.5) node [midway,left] {$c$};
% bac - ba
\draw[color=snow] (1.5,3.6)-- (1.5,4.4) node [midway,left,color=snow] {$c$};
% acba - abcba
\draw[color=snow] (3.5,0.5)-- (3.5,1.5) node [midway,right] {$c$};
% abac - aba
\draw[color=snow] (3.5,3.5)-- (3.5,4.5) node [midway,right] {$c$};
% acb - abcb
\draw[color=snow] (4.5,0.5)-- (4.5,1.5) node [midway,left] {$c$};
% abc - ab
\draw[color=snow] (4.5,3.5)-- (4.5,4.4) node [midway,left,color=snow] {$c$};
% bacba - abacba
\draw[color=snow] (2,2)-- (3,2) node [midway,above] {$c$};
% bacb - abacb
\draw[color=snow] (2,3)-- (2.9,3) node [midway,below,color=snow] {$c$};

% GENERATOR b
% e - b
\draw[color=snow] (0.1,4.9)-- (0.5,4.5) node [color=snow,midway,below] {$b$};
% bac - bacb
\draw[color=snow] (1.6,3.4)-- (1.9,3.1) node [at end,above,color=snow] {$b$};
% bacba - bcba
\draw[color=snow] (3,2)-- (3.5,1.5) node [near end,above] {$b$};
% c - cb
\draw[color=snow] (0,0)-- (0.5,0.5) node [midway,above] {$b$};
% bcba - bacba
\draw[color=snow] (1.5,1.5)-- (2,2) node [near start,above] {$b$};
% abacb - abac
\draw[color=snow] (3,3)-- (3.4,3.4) node [near start,above,color=snow] {$b$};

% a - ab
\draw[color=snow] (4.9,4.9)-- (4.5,4.5) node [midway,below,color=snow] {$b$};
% ac - acb
\draw[color=snow] (5,0)-- (4.5,0.5) node [midway,above] {$b$};

% cba - acba
\draw[color=snow] (1.5,0.5)-- (3.5,0.5) node [midway,below] {$b$};
% ba - aba
\draw[color=snow] (1.5,4.5)-- (3.5,4.5) node [midway,above] {$b$};
% bcb - bc
\draw[color=snow] (0.5,1.5)-- (0.5,3.5) node [midway,right] {$b$};
% abcb - abc
\draw[color=snow] (4.5,1.5)-- (4.5,3.5) node [midway,left] {$b$};

%*************** nodos de la grafica *********************************

% e_A - 1234
\fill [color=gold ] (0,5) circle (3.5pt);
\draw[color=snow] (0.2,5.2) node {$\overline{e_\mathcal{A}}$};

% a - 2134
\fill [color=gold] (5,5) circle (3.5pt);
\draw[color=snow] (4.8,5.2) node {$\overline{a}$};

% c - 1243
\fill [color=gold] (0,0) circle (3.5pt);
\draw[color=snow] (0.2,-0.2) node {$\overline{c}$};

% ac - 2143
\fill [color=gold] (5,0) circle (3.5pt);
\draw[color=snow] (4.8,-0.2) node {$\overline{ac}$};

%%%%%%%%%%%%%%%%%%%%%%%%%%%%%%%%%%%%%%%%%

% b - 1324
\fill [color=gold ] (0.5,4.5) circle (3.5pt);
\draw[color=snow] (0.7,4.7) node {$\overline{b}$};

% ab - 2314
\fill [color=gold] (4.5,4.5) circle (3.5pt);
\draw[color=snow] (4.3,4.7) node {$\overline{ab}$};

% cb - 1423
\fill [color=gold] (0.5,0.5) circle (3.5pt);
\draw[color=snow] (0.7,0.3) node {$\overline{cb}$};

% acb - 2413
\fill [color=gold] (4.5,0.5) circle (3.5pt);
\draw[color=snow] (4.3,0.3) node {$\overline{acb}$};

%%%%%%%%%%%%%%%%%%%%%%%%%%%%%%%%%%%%%%%%%

% ba - 3124
\fill [color=gold] (1.5,4.5) circle (3.5pt);
\draw[color=snow] (1.8,4.3) node {$\overline{ba}$};

% aba - 3214
\fill [color=gold] (3.5,4.5) circle (3.5pt);
\draw[color=snow] (3.1,4.3) node {$\overline{aba}$};

% cba - 4123
\fill [color=gold] (1.5,0.5) circle (3.5pt);
\draw[color=snow] (1.85,0.7) node {$\overline{cba}$};

% acba - 4213
\fill [color=gold] (3.5,0.5) circle (3.5pt);
\draw[color=snow] (3.05,0.7) node {$\overline{acba}$};

%%%%%%%%%%%%%%%%%%%%%%%%%%%%%%%%%%%%%%%%%

% bc - 1324
\fill [color=gold ] (0.5,3.5) circle (3.5pt);
\draw[color=snow] (0.7,3.3) node {$\overline{bc}$};

% abc - 2341
\fill [color=gold] (4.5,3.5) circle (3.5pt);
\draw[color=snow] (4.2,3.3) node {$\overline{abc}$};

% bcb - 1423
\fill [color=gold] (0.5,1.5) circle (3.5pt);
\draw[color=snow] (0.8,1.7) node {$\overline{bcb}$};

% abcb - 2431
\fill [color=gold] (4.5,1.5) circle (3.5pt);
\draw[color=snow] (4.1,1.7) node {$\overline{abcb}$};

%%%%%%%%%%%%%%%%%%%%%%%%%%%%%%%%%%%%%%%%%

% bac - 3124
 \fill [color=gold ] (1.5,3.5) circle (3.5pt);
\draw[color=snow] (1.9,3.6) node {$\overline{bac}$};

% abac - 3241
\fill [color=gold] (3.5,3.5) circle (3.5pt);
\draw[color=snow] (3,3.6) node {$\overline{abac}$};

% bcba - 4132
\fill [color=gold] (1.5,1.5) circle (3.5pt);
\draw[color=snow] (1.95,1.4) node {$\overline{bcba}$};

% abcba - 4231
\fill [color=gold] (3.5,1.5) circle (3.5pt);
\draw[color=snow] (2.95,1.4) node {$\overline{abcba}$};

%%%%%%%%%%%%%%%%%%%%%%%%%%%%%%%%%%%%%%%%%

% abacb - 3421
\fill [color=gold] (3,3) circle (3.5pt);
\draw[color=snow] (3.5,2.85) node {$\overline{abacb}$};

% bacb - 3412
\fill [color=gold ] (2,3) circle (3.5pt);
\draw[color=snow] (1.6,2.85) node {$\overline{bacb}$};

% abacba - 4312
\fill [color=gold] (3,2) circle (3.5pt);
\draw[color=snow] (3.6,2.2) node {$\overline{abacba}$};

% bacba - 4321
\fill [color=gold] (2,2) circle (3.5pt);
\draw[color=snow] (1.45,2.2) node {$\overline{bacba}$};

\end{scriptsize}
\end{tikzpicture}
    \caption{$BS(4)$ with  links and node labelled. Each Link is labelled with a letter in Eq. (\ref{eq:mapGenAlpBS}) representing its corresponding generator. Meanwhile, each node is labelled with a word in Eq. (\ref{eq:lan-B4}) representing the \textit{shortLex} path from $\overline{e_\mathcal{A}}$ to it.}
    \label{fig:BS4_nodeEdgeLabelled}
\end{figure}

\subsection{Path computation algorithms}
\label{sec:path_algorithms}

The word-processing approach presented in this section allows to compute paths in CGs as follows.
Note that there is a path between every two nodes $\overline{u}$ and $\overline{v}$ given by $\widehat{u^{-1}v}$ \footnote{The path $\widehat{u^{-1}v}$ goes from $\overline{u}$ to  $\overline{e_\mathcal{A}}$ and from  $\overline{e_\mathcal{A}}$ to  $\overline{v}$. Taking Fig. \ref{fig:BS4_nodeEdgeLabelled} as example, the path $\widehat{(bc)^{-1}ab}=\widehat{cbab}$ goes from $\overline{bc}$ to $\overline{e_\mathcal{A}}$ and from $\overline{e_\mathcal{A}}$ to $\overline{ab}$.}. From the definition of equivalent words, every word $w=_\mathcal{G}u^{-1}v$ represents a path between $\overline{u}$ and $\overline{v}$, see Section \ref{sec:words_as_paths}. Therefore the problem of compute the paths between    $\overline{u}$ and $\overline{v}$ is equivalent to compute words equivalent to $u^{-1}v$. Authors in \cite{AGUIRREGUERRERO2019218} present a set of path computation algorithms that take as input the word $u^{-1}v$, and then compute the words equivalent to $u^{-1}v$ as in shown in Table \ref{tb:algorithms}.
The paths are computed in an ordered manner from the shortest to the largest one by using an automaton called the \textit{Word-Difference Automaton} (WDA). The WDA encodes the topological structure of the CG and recognises words representing the same pair of nodes \cite[Section 13.2.2]{derekHCGT}.

\begin{table}[t!]
\centering
\caption{Path computation algorithms, its computed words and paths \cite{AGUIRREGUERRERO2019218}.}
	\label{tb:algorithms}
\begin{tabular}[center]{|m{0.9cm}|m{4cm}|m{2.6cm}|}
\cline{2-3}
\rowcolor{gray!30}\multicolumn{1}{m{0.6cm}|}{\cellcolor{white}}&&\multicolumn{1}{c|}{\multirow{1}{*}{\textbf{Paths represented by}}}\\
\rowcolor{gray!30}\multicolumn{1}{m{0.6cm}|}{\cellcolor{white}}&\multicolumn{1}{c|}{\multirow{-2}{*}{\textbf{Computed words}}}&\multicolumn{1}{c|}{\multirow{1}{*}{\textbf{the computed words}}}\\
\cline{2-3}\noalign{\smallskip}
%\multirow{1}{*}{$\mathcal{G}=<S|R>$}   &\multirow{1}{*}{Finitely presented group with generating set $S$ and}\\
%\multirow{1}{*}{}   &\multirow{1}{*}{set of relators $R$.}\\
\hline
\cellcolor{gray!30}\multirow{1}{*}{\textbf{Alg. A}} &\multirow{1}{*}{The \textit{shortLex} word.} &\multirow{1}{*}{The shortest path.}\\
\hline
%& \\
\cellcolor{gray!30}\multirow{1}{*}{\textbf{Alg. B}}&\multirow{1}{*}{The $K$-\textit{shortLex} words.} &\multirow{1}{*}{The $K$-shortest paths.}\\
\hline
\cellcolor{gray!30}\multirow{1}{*}{}&\multirow{1}{*}{The \textit{shortLex} words without}   &\multirow{1}{*}{The shortest disjoint}\\
\cellcolor{gray!30}\multirow{-2}{*}{\textbf{Alg. C}}  &\multirow{1}{*}{equivalent substrings to each other.} &\multirow{1}{*}{paths.}\\
\hline
\cellcolor{gray!30}\multirow{1}{*}{}  &\multirow{2}{*}{The \textit{shortLex} word without sub-}&\multirow{1}{*}{The shortest path}\\
\cellcolor{gray!30}&\multirow{1}{*}{}&\multirow{1}{*}{ avoiding a set of}\\
\cellcolor{gray!30}\multirow{-3}{*}{\textbf{Alg. D}} &\multirow{-2}{*}{strings equivalent to a set of words.} &\multirow{1}{*}{nodes and links.}\\
\hline
\end{tabular}\\
\end{table}

As is explained in the section below, the path computation algorithms presented in \cite{AGUIRREGUERRERO2019218} are used in the WPR to compute the routing paths. 
The following lemma states the time complexity of these algorithms.

\begin{lemma}
\label{lem:pc_alg}
The algorithms presented in \cite{AGUIRREGUERRERO2019218} are able to compute the $K$-shortest paths, the shortest disjoint paths and the set of paths avoiding nodes and edges in time $O(K\ell|Diff|)$, where $K$ denotes the number of computed paths, $\ell$ denotes the length of the largest computed path and $|Diff|$ denotes the size of the WDA. \cite[Colloraries 5-7 ]{AGUIRREGUERRERO2019218}.
\end{lemma}